\documentclass[letterpaper]{article}

\usepackage[utf8]{inputenc}
\usepackage[T1]{fontenc}
\usepackage[spanish]{babel}
\usepackage{lmodern}
\usepackage{amsmath}
\usepackage[margin=0.3cm]{geometry}

\title{Resumen Parcial 1 Sistemas de Control II}
\author{Emerson Warhman}
\begin{document}
\maketitle

% Aquí va el contenido del resumen
\section*{Forma estandar Sistema de segundo Orden}
La forma estándar de un sistema de control de segundo orden con realimentación negativa unitaria es:

\[ T(s) = \frac{\omega_n^2}{s^2 + 2\zeta \omega_n s + \omega_n^2} \]

Los polos del sistema son:

\[ s = -\zeta \omega_n \pm \omega_n \sqrt{\zeta^2 - 1} \]

Las condiciones de amortiguamiento según el valor de $\zeta$ son:

\begin{itemize}
    \item Si \(\zeta > 1\): Sobreamortiguado
    \item Si \(\zeta = 1\): Amortiguamiento crítico
    \item Si \(0 < \zeta < 1\): Subamortiguado
    \item Si \(\zeta = 0\): Sin amortiguamiento
\end{itemize}

\subsection*{Características Transitorias}

Para sistemas subamortiguados (\(0 < \zeta < 1\)):

\begin{itemize}
    \item Tiempo de alza: \( t_r = \frac{1}{\omega_n} \left( \pi - \tan^{-1} \frac{\sqrt{1-\zeta^2}}{\zeta} \right) \)
    \item Tiempo pico: \( t_p = \frac{\pi}{\omega_n \sqrt{1-\zeta^2}} \)
    \item Sobrepico: \( M_p = e^{-\frac{\zeta \pi}{\sqrt{1-\zeta^2}}} \)
    \item Tiempo de establecimiento (criterio 2\%): \( t_s = \frac{4}{\zeta \omega_n} \)
    \item Frecuencia amortiguada: \(\omega_d = \omega_n \sqrt{1-\zeta^2}\)
\end{itemize}

\section*{Estabilidad de un Sistema}

\subsection*{Criterio de Jury para Sistemas Discretos}

El criterio de Jury determina si todos los polos de un sistema discreto están dentro del círculo unitario en el plano z. Los pasos para aplicar el criterio son:

\begin{enumerate}
    \item Obtén el polinomio característico en la forma: \( z^n + a_{n-1} z^{n-1} + \dots + a_1 z + a_0 = 0 \).
    \item Verifica la condición necesaria: \( |a_0| < 1 \). Si no se cumple, el sistema es inestable.
    \item Construye la tabla de Jury con 2n filas:
          - Fila 1: Coeficientes del polinomio de mayor a menor grado $(1, a_{n-1}, ..., a_0)$.
          - Fila 2: Coeficientes del polinomio de menor a mayor grado $(a_0, a_1, ..., 1)$.
          - Para filas subsiguientes, calcula los elementos usando determinantes de matrices 2x2 formadas por las dos filas anteriores.
    \item Verifica que todos los elementos de la primera columna de la tabla sean positivos. Si es así, el sistema es estable.
    \item Si algún elemento de la primera columna es cero o negativo, el sistema tiene polos en o fuera del círculo unitario.
\end{enumerate}

\subsection*{Criterio de Routh para Sistemas Continuos}

El criterio de Routh determina si todos los polos de un sistema continuo están en el semiplano izquierdo. Los pasos para aplicar el criterio son:

\begin{enumerate}
    \item Expande la ecuación característica en forma descendente: \( s^n + a_{n-1} s^{n-1} + \dots + a_1 s + a_0 = 0 \).
    \item Verifica que todos los coeficientes sean positivos. Si no todos son positivos, o si alguno es cero, o si hay raíces imaginarias, el sistema es inestable.
    \item Ordena los coeficientes: Primera fila del arreglo de Routh con coeficientes de potencias pares de s $(s^n, s^{n-2}, ...)$. Segunda fila con coeficientes de potencias impares $(s^{n-1}, s^{n-3}$, ...).
    \item Para filas subsiguientes, calcula cada elemento usando: \( -\frac{1}{a_{k-1,1}} \det \begin{pmatrix} a_{k-2,1} & a_{k-2,j+1} \\ a_{k-1,1} & a_{k-1,j+1} \end{pmatrix} \).
    \item Verifica que todos los elementos de la primera columna del arreglo sean positivos. Si es así, el sistema es estable.
    \item Si hay un cero en la primera columna, reemplázalo con un epsilon pequeño y analiza el límite cuando epsilon tiende a cero.
    \item El número de cambios de signo en la primera columna indica el número de polos en el semiplano derecho.
    \item Si todos los miembros de una fila impar son 0, se deriva la fila superior y dicha derivada se utiliza en la fila donde daba los ceros, a partir de ahí se continua con el procedimiento habitual.
\end{enumerate}

\section*{Lugar Geométrico de las Raíces}

El lugar geométrico de las raíces es un método gráfico para analizar cómo cambian los polos del sistema cerrado al variar un parámetro, generalmente la ganancia K. Los pasos para trazar el lugar geométrico de las raíces son:

\begin{enumerate}
    \item Determina la función de transferencia en lazo abierto \( G(s)H(s) \).
    \item Identifica los polos y ceros de \( G(s)H(s) \).
    \item Dibuja los segmentos del eje real que pertenecen al lugar geométrico. Usar la regla de paridad de raíces a la derecha de un punto.
    \item Determina las asíntotas: número de asíntotas, ángulos y punto de intersección con el eje real.
          \begin{itemize}
              \item $\theta_{asint}=\frac{180 (2k+1)}{N_p - N_z}$
              \item $\sigma_{asint}=\frac{\sum P_j - \sum Z_i  }{N_p - N_z}$
          \end{itemize}
    \item Encuentra los puntos de ruptura y entrada resolviendo la ecuación característica.
          \begin{itemize}
              \item Si el LGR está entre dos polos en $\sigma$, existe al menos un punto de ruptura ahí.
              \item Si el LGR está entre dos ceros en $\sigma$, existe al menos un punto de ingreso ahí.
              \item Si el LGR está entre un cero y un polo pueden no existir ni puntos de ruptura ni de ingreso. o pueden haber ambos.
          \end{itemize}
          A partir de la E.C se despeja K y se deriva.
          \(\frac{dK}{ds}=-\frac{1 + GH(s)}{G(s)}=0\).\\Se hallan las raíces de dicho polinomio y se evaluán en $K(s_i)$.\\ Si $K(s_i)> 0$ entonces sí es un punto de ruptura o de ingreso.
    \item determinar el ángulo de sálida y ángulo de llegada.
          \[\theta_{sal}=180 - \sum \angle P + \sum \angle Z\]
          \[\theta_{ent}=180 - \sum\angle Z + \sum\angle P \]
    \item Determinar las intersecciones con el eje imaginario usando el criterio de Routh o similar.
    \item Dibuja el lugar conectando los polos y ceros, siguiendo las reglas de magnitud y ángulo. Tomando una serie de puntos de pruebas en las cercanias del origen del plano.
\end{enumerate}

\section*{Compensadores}

Los compensadores se utilizan para mejorar el rendimiento de los sistemas de control, como estabilidad, velocidad de respuesta y error en estado estacionario.

\begin{itemize}
    \item \textbf{Compensador de adelanto (lead)}: Aumenta la velocidad de respuesta y el margen de fase. Función de transferencia: \( G_c(s) = K \frac{s + z}{s + p} \) donde \( z < p \).
    \item \textbf{Compensador de atraso (lag)}: Reduce el error en estado estacionario. Función de transferencia: \( G_c(s) = K \frac{s + z}{s + p} \) donde \( z > p \).
    \item \textbf{Compensador lead-lag}: Combina adelanto y atraso para mejorar múltiples características.
    \item \textbf{Controlador PID}: Proporcional-Integral-Derivativo. Función: \( G_c(s) = K_p + \frac{K_i}{s} + K_d s \), donde \( K_p \) para proporcional, \( K_i \) para integral, \( K_d \) para derivativo.
\end{itemize}

\subsection*{Diseño de un Compensador de Adelanto usando LGR}

\[
    G_c(s) = Kc\cdot\alpha\frac{Ts + 1}{\alpha Ts + 1}= K_c\frac{s+ 1/T }{S + 1 / \alpha T}
\]

Los pasos para diseñar un compensador de adelanto usando el lugar geométrico de las raíces son:

\begin{enumerate}
    \item Especifica las características deseadas del sistema cerrado, como el factor de amortiguamiento \(\zeta\) y la frecuencia natural \(\omega_n\).
    \item Traza el LGR del sistema sin compensar y determina la ganancia máxima para estabilidad.
    \item Determina la ubicación deseada de los polos dominantes en el plano s.
    \item Calcula el ángulo de fase adicional requerido para mover el LGR a la ubicación deseada.
    \item Elige la ubicación del polo del compensador: \( z = -\frac{1}{\alpha T} \), donde \(\alpha < 1\) y T se elige para el máximo adelanto.
    \item Elige la ubicación del cero del compensador: \( p = -\frac{1}{T} \), con \( T = \frac{1}{\omega_m \sqrt{\alpha}} \), donde \(\omega_m\) es la frecuencia de máxima fase.
    \item Ajusta la ganancia K para que el LGR pase por la ubicación deseada.
\end{enumerate}

\subsection*{Diseño de un Compensador de Atraso usando LGR}

\[
    G_c(s) = K_c \cdot \beta \frac{Ts + 1}{\beta Ts + 1} = K_c \frac{s + 1/T}{s + 1/(\beta T)}
\]

Los pasos para diseñar un compensador de atraso usando el lugar geométrico de las raíces son:

\begin{enumerate}
    \item Especifica las características deseadas del sistema cerrado, como el factor de amortiguamiento \(\zeta\), la frecuencia natural \(\omega_n\) y el error en estado estacionario.
    \item Traza el LGR del sistema sin compensar y determina la ganancia K que coloca los polos en la ubicación deseada.
    \item Verifica el error en estado estacionario. Si no cumple con las especificaciones, diseña un compensador de atraso.
    \item Elige la ubicación del polo del compensador: \( p = -\frac{1}{\beta T} \), donde \(\beta > 1\) y T es grande para que el polo esté lejos a la izquierda.
    \item Elige la ubicación del cero del compensador: \( z = -\frac{1}{T} \), con \( z \) cerca del polo para minimizar el cambio de fase.
    \item Ajusta la ganancia \( K_c \) para que el LGR del sistema compensado pase por la ubicación deseada de los polos.
\end{enumerate}

\section*{Sistemas muestreados}


\[X^*(s) = \sum_{k=0}^{\infty}x(kT)e^{-skT} \]
Si se define $z=e^{sT}$ entonces $s=\frac{1}{T}ln(z)$
\[X(z) = \sum_{k=0}^{\infty}x(kT)z^{-k} \]

$X(z)$ es la transformada Z de $x(kT)$

Serie geométrica básica:
\begin{center}
    \(\sum_{k=0}^\infty r^k= \frac{1}{1- r} \), para $|r| < 1$.
\end{center}


\subsection*{Transformada Z inversa}

\begin{itemize}
    \item Método directo:
          \[X(z) =x(0) + x(1)z^{-1} + x(2)z^{-2} + ...\]
          Se obtiene dividiendo polinomio del numerador entre polinomio del denominador.
    \item Fracciones parciales
          \[X(z) = Z \{L^{-1}\{X(s)\}_{kT}\}\]
    \item Tabla de transformaciones
\end{itemize}

\subsection*{Circuitos para la retención de datos}

\begin{itemize}
    \item Retentor de orden cero
          \[Gh_0(s) = \frac{1- e^{-sT}}{s}\]
    \item Retentor de orden uno
          \[Gh_1(s) = \left(\frac{1- e^{-sT}}{s}\right)^2 \frac{Ts +1 }{T}\]
\end{itemize}
\subsection*{Función de transferencia de pulso}

La función de transferencia de pulso para un sistema continuo con retenedor de orden cero es:

\[ G(z) = (1 - z^{-1}) \mathcal{Z} \left\{ \frac{G(s)}{s} \right\} \]

\section*{Respuesta transitoria de los sistemas discretos}
Dados los polos de un sistema en tiempo continuo

\[ s = -\zeta \omega_n \pm \omega_n \sqrt{\zeta^2 - 1} \]

Dado que \(z=e^{sT}\)

tenemos

\[z=e^{-\zeta\omega_nT+j\omega_nT\sqrt{1-\zeta^2}}\]
por tanto

\begin{center}
    \(|z| = e^{-T\zeta\omega_n}\) \hfil \(\angle{z}= \omega_dT\)
\end{center}

al hacer \(T=\frac{2\pi}{\omega_s}\)

\begin{center}
    \(|z| = e^{\frac{-2\pi\zeta\omega_d }{\sqrt{1-\zeta^2}\omega_s }}\) \hfil \(\angle{z}= \frac{2\pi\omega_dT}{\omega_s}\)
\end{center}




\end{document}
