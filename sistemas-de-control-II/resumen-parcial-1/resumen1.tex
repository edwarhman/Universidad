\documentclass[letterpaper]{article}

\usepackage[utf8]{inputenc}
\usepackage[T1]{fontenc}
\usepackage[spanish]{babel}
\usepackage{lmodern}
\usepackage{amsmath}
\usepackage[margin=0.3cm]{geometry}

\title{Resumen Parcial 1 Sistemas de Control II}
\author{Emerson Warhman}
\begin{document}
\maketitle

% Aquí va el contenido del resumen
\section*{Forma estandar Sistema de segundo Orden}
La forma estándar de un sistema de control de segundo orden con realimentación negativa unitaria es:

\[ T(s) = \frac{\omega_n^2}{s^2 + 2\zeta \omega_n s + \omega_n^2} \]

Los polos del sistema son:

\[ s = -\zeta \omega_n \pm \omega_n \sqrt{\zeta^2 - 1} \]

Las condiciones de amortiguamiento según el valor de $\zeta$ son:

\begin{itemize}
    \item Si \(\zeta > 1\): Sobreamortiguado
    \item Si \(\zeta = 1\): Amortiguamiento crítico
    \item Si \(0 < \zeta < 1\): Subamortiguado
    \item Si \(\zeta = 0\): Sin amortiguamiento
\end{itemize}

\subsection*{Características Transitorias}

Para sistemas subamortiguados (\(0 < \zeta < 1\)):

\begin{itemize}
    \item Tiempo de alza: \( t_r = \frac{1}{\omega_n} \left( \pi - \tan^{-1} \frac{\sqrt{1-\zeta^2}}{\zeta} \right) \)
    \item Tiempo pico: \( t_p = \frac{\pi}{\omega_n \sqrt{1-\zeta^2}} \)
    \item Sobrepico: \( M_p = e^{-\frac{\zeta \pi}{\sqrt{1-\zeta^2}}} \)
    \item Tiempo de establecimiento (criterio 2\%): \( t_s = \frac{4}{\zeta \omega_n} \)
    \item Frecuencia amortiguada: \(\omega_d = \omega_n \sqrt{1-\zeta^2}\)
\end{itemize}

\section*{Estabilidad de un Sistema}

\subsection*{Criterio de Jury para Sistemas Discretos}

El criterio de Jury determina si todos los polos de un sistema discreto están dentro del círculo unitario en el plano z. Los pasos para aplicar el criterio son:

\begin{enumerate}
    \item Obtén el polinomio característico en la forma: \( z^n + a_{n-1} z^{n-1} + \dots + a_1 z + a_0 = 0 \).
    \item Verifica la condición necesaria: \( |a_0| < 1 \). Si no se cumple, el sistema es inestable.
    \item Construye la tabla de Jury con 2n filas:
          - Fila 1: Coeficientes del polinomio de mayor a menor grado $(1, a_{n-1}, ..., a_0)$.
          - Fila 2: Coeficientes del polinomio de menor a mayor grado $(a_0, a_1, ..., 1)$.
          - Para filas subsiguientes, calcula los elementos usando determinantes de matrices 2x2 formadas por las dos filas anteriores.
    \item Verifica que todos los elementos de la primera columna de la tabla sean positivos. Si es así, el sistema es estable.
    \item Si algún elemento de la primera columna es cero o negativo, el sistema tiene polos en o fuera del círculo unitario.
\end{enumerate}

\subsection*{Criterio de Routh para Sistemas Continuos}

El criterio de Routh determina si todos los polos de un sistema continuo están en el semiplano izquierdo. Los pasos para aplicar el criterio son:

\begin{enumerate}
    \item Expande la ecuación característica en forma descendente: \( s^n + a_{n-1} s^{n-1} + \dots + a_1 s + a_0 = 0 \).
    \item Verifica que todos los coeficientes sean positivos. Si no todos son positivos, o si alguno es cero, o si hay raíces imaginarias, el sistema es inestable.
    \item Ordena los coeficientes: Primera fila del arreglo de Routh con coeficientes de potencias pares de s $(s^n, s^{n-2}, ...)$. Segunda fila con coeficientes de potencias impares $(s^{n-1}, s^{n-3}$, ...).
    \item Para filas subsiguientes, calcula cada elemento usando: \( -\frac{1}{a_{k-1,1}} \det \begin{pmatrix} a_{k-2,1} & a_{k-2,j+1} \\ a_{k-1,1} & a_{k-1,j+1} \end{pmatrix} \).
    \item Verifica que todos los elementos de la primera columna del arreglo sean positivos. Si es así, el sistema es estable.
    \item Si hay un cero en la primera columna, reemplázalo con un epsilon pequeño y analiza el límite cuando epsilon tiende a cero.
    \item El número de cambios de signo en la primera columna indica el número de polos en el semiplano derecho.
    \item Si todos los miembros de una fila impar son 0, se deriva la fila superior y dicha derivada se utiliza en la fila donde daba los ceros, a partir de ahí se continua con el procedimiento habitual.
\end{enumerate}

\section*{Lugar Geométrico de las Raíces}

El lugar geométrico de las raíces es un método gráfico para analizar cómo cambian los polos del sistema cerrado al variar un parámetro, generalmente la ganancia K. Los pasos para trazar el lugar geométrico de las raíces son:

\begin{enumerate}
    \item Determina la función de transferencia en lazo abierto \( G(s)H(s) \).
    \item Identifica los polos y ceros de \( G(s)H(s) \).
    \item Dibuja los segmentos del eje real que pertenecen al lugar geométrico. Usar la regla de paridad de raíces a la derecha de un punto.
    \item Determina las asíntotas: número de asíntotas, ángulos y punto de intersección con el eje real.
          \begin{itemize}
              \item $\theta_{asint}=\frac{180 (2k+1)}{N_p - N_z}$
              \item $\sigma_{asint}=\frac{\sum P_j - \sum Z_i  }{N_p - N_z}$
          \end{itemize}
    \item Encuentra los puntos de ruptura y entrada resolviendo la ecuación característica.
          \begin{itemize}
              \item Si el LGR está entre dos polos en $\sigma$, existe al menos un punto de ruptura ahí.
              \item Si el LGR está entre dos ceros en $\sigma$, existe al menos un punto de ingreso ahí.
              \item Si el LGR está entre un cero y un polo pueden no existir ni puntos de ruptura ni de ingreso. o pueden haber ambos.
          \end{itemize}
          A partir de la E.C se despeja K y se deriva.
          \(\frac{dK}{ds}=-\frac{1 + GH(s)}{G(s)}=0\).\\Se hallan las raíces de dicho polinomio y se evaluán en $K(s_i)$.\\ Si $K(s_i)> 0$ entonces sí es un punto de ruptura o de ingreso.
    \item determinar el ángulo de sálida y ángulo de llegada.
          \[\theta_{sal}=180 - \sum \angle P + \sum \angle Z\]
          \[\theta_{ent}=180 - \sum\angle Z + \sum\angle P \]
    \item Determinar las intersecciones con el eje imaginario usando el criterio de Routh o similar.
    \item Dibuja el lugar conectando los polos y ceros, siguiendo las reglas de magnitud y ángulo. Tomando una serie de puntos de pruebas en las cercanias del origen del plano.
\end{enumerate}

\section*{Compensadores}

Los compensadores se utilizan para mejorar el rendimiento de los sistemas de control, como estabilidad, velocidad de respuesta y error en estado estacionario.

\begin{itemize}
    \item \textbf{Compensador de adelanto (lead)}: Aumenta la velocidad de respuesta y el margen de fase. Función de transferencia: \( G_c(s) = K \frac{s + z}{s + p} \) donde \( z < p \).
    \item \textbf{Compensador de atraso (lag)}: Reduce el error en estado estacionario. Función de transferencia: \( G_c(s) = K \frac{s + z}{s + p} \) donde \( z > p \).
    \item \textbf{Compensador lead-lag}: Combina adelanto y atraso para mejorar múltiples características.
    \item \textbf{Controlador PID}: Proporcional-Integral-Derivativo. Función: \( G_c(s) = K_p + \frac{K_i}{s} + K_d s \), donde \( K_p \) para proporcional, \( K_i \) para integral, \( K_d \) para derivativo.
\end{itemize}

\subsection*{Diseño de un Compensador de Adelanto usando LGR}

\[
    G_c(s) = K_c \cdot \alpha \frac{T s + 1}{\alpha T s + 1} = K_c \frac{s + 1/T}{s + 1/(\alpha T)}
\]

Los pasos para diseñar un compensador de adelanto usando el lugar geométrico de las raíces son:

\begin{enumerate}
    \item Especifica las características deseadas del sistema cerrado, como el factor de amortiguamiento \(\zeta\) y la frecuencia natural \(\omega_n\).
    \item Traza el LGR del sistema sin compensar y determina la ganancia máxima para estabilidad.
    \item Determina la ubicación deseada de los polos dominantes en el plano s.
    \item Calcula el ángulo de fase adicional requerido para mover el LGR a la ubicación deseada.
    \item Elige la ubicación del polo del compensador: \( z = -\frac{1}{\alpha T} \), donde \(\alpha < 1\) y T se elige para el máximo adelanto.
    \item Elige la ubicación del cero del compensador: \( p = -\frac{1}{T} \), con \( T = \frac{1}{\omega_m \sqrt{\alpha}} \), donde \(\omega_m\) es la frecuencia de máxima fase.
    \item Ajusta la ganancia K para que el LGR pase por la ubicación deseada.
\end{enumerate}

\subsection*{Diseño de un Compensador de Atraso usando LGR}

\[
    G_c(s) = K_c \cdot \beta \frac{Ts + 1}{\beta Ts + 1} = K_c \frac{s + 1/T}{s + 1/(\beta T)}
\]

Los pasos para diseñar un compensador de atraso usando el lugar geométrico de las raíces son:

\begin{enumerate}
    \item Especifica las características deseadas del sistema cerrado, como el factor de amortiguamiento \(\zeta\), la frecuencia natural \(\omega_n\) y el error en estado estacionario.
    \item Traza el LGR del sistema sin compensar y determina la ganancia K que coloca los polos en la ubicación deseada.
    \item Verifica el error en estado estacionario. Si no cumple con las especificaciones, diseña un compensador de atraso.
    \item Elige la ubicación del polo del compensador: \( p = -\frac{1}{\beta T} \), donde \(\beta > 1\) y T es grande para que el polo esté lejos a la izquierda.
    \item Elige la ubicación del cero del compensador: \( z = -\frac{1}{T} \), con \( z \) cerca del polo para minimizar el cambio de fase.
    \item Ajusta la ganancia \( K_c \) para que el LGR del sistema compensado pase por la ubicación deseada de los polos.
\end{enumerate}

\section*{Sistemas muestreados}


\[X^*(s) = \sum_{k=0}^{\infty}x(kT)e^{-skT} \]
Si se define $z=e^{sT}$ entonces $s=\frac{1}{T}ln(z)$
\[X(z) = \sum_{k=0}^{\infty}x(kT)z^{-k} \]

$X(z)$ es la transformada Z de $x(kT)$

Serie geométrica básica:
\begin{center}
    \(\sum_{k=0}^\infty r^k= \frac{1}{1- r} \), para $|r| < 1$.
\end{center}



\subsection*{Transformada Z inversa}

\begin{itemize}
    \item Método directo:
          \[X(z) =x(0) + x(1)z^{-1} + x(2)z^{-2} + ...\]
          Se obtiene dividiendo polinomio del numerador entre polinomio del denominador.
    \item Fracciones parciales
          \[X(z) = Z \{L^{-1}\{X(s)\}_{kT}\}\]
    \item Tabla de transformaciones
\end{itemize}

\subsection*{Circuitos para la retención de datos}

\begin{itemize}
    \item Retentor de orden cero
          \[Gh_0(s) = \frac{1- e^{-sT}}{s}\]
    \item Retentor de orden uno
          \[Gh_1(s) = \left(\frac{1- e^{-sT}}{s}\right)^2 \frac{Ts +1 }{T}\]
\end{itemize}
\subsection*{Función de transferencia de pulso}

La función de transferencia de pulso para un sistema continuo con retenedor de orden cero es:

\[ G(z) = (1 - z^{-1}) \mathcal{Z} \left\{ \frac{G(s)}{s} \right\} \]

\section*{Respuesta transitoria de los sistemas discretos}
Dados los polos de un sistema en tiempo continuo

\[ s = -\zeta \omega_n \pm \omega_n \sqrt{\zeta^2 - 1} \]

Dado que \(z=e^{sT}\)

tenemos

\[z=e^{-\zeta\omega_nT+j\omega_nT\sqrt{1-\zeta^2}}\]
por tanto

\begin{center}
    \(|z| = e^{-T\zeta\omega_n}\) \hfil \(\angle{z}= \omega_dT\)
\end{center}

al hacer \(T=\frac{2\pi}{\omega_s}\)

\begin{center}
    \(|z| = e^{\frac{-2\pi\zeta\omega_d }{\sqrt{1-\zeta^2}\omega_s }}\) \hfil \(\angle{z}= \frac{2\pi\omega_dT}{\omega_s}\)
\end{center}




\section*{Variables de Estado}

Las variables de estado son un conjunto de variables que describen completamente el estado interno de un sistema dinámico. En la representación en espacio de estados, un sistema se describe mediante ecuaciones diferenciales de primer orden:

\[\dot{x}(t) = A x(t) + B u(t)\]

\[y(t) = C x(t) + D u(t)\]

Donde:

\begin{itemize}
    \item \(x(t)\): Vector de estado, que contiene las variables que describen el estado del sistema en el tiempo t.
    \item \(u(t)\): Vector de entrada o señal de control aplicada al sistema.
    \item \(y(t)\): Vector de salida o respuesta del sistema.
    \item \(A\): Matriz del sistema, que describe la evolución del estado (la dinámica) interno del sistema.
    \item \(B\): Matriz de control, que relaciona cómo la entrada afecta al estado.
    \item \(C\): Matriz de salida, que relaciona el estado con la salida medida.
    \item \(D\): Matriz de transmisión directa, que describe la influencia directa de la entrada en la salida (usualmente cero en muchos sistemas).
\end{itemize}

Notamos que esta representación permite múltiples entradas y múltiples salidas. Además las matrices pueden ser funciones de tiempo
En el caso de un sistema con una entrada y una salida tendremos (Sistemas Componsados)
Notamos que en la configuración clásica solo se realimenta la salida.
En la configuración moderna del un lazo de control automático se realimentan todas las
variables de estado

\subsection*{Relación entre la representación en espacio de estados y la función de transferencia}

La función de transferencia G(s) se puede obtener a partir de las matrices de estado de la siguiente manera:

\[ G(s) = C (sI - A)^{-1} B + D \]

Donde I es la matriz identidad de tamaño n x n, y $(sI - A)^{-1}$ se conoce como la matriz resolvente. Los polos de la función de transferencia son los valores propios de la matriz A.

\section*{Controlabilidad y observabilidad}

\begin{itemize}
    \item Controlabilidad: un sistema es controlable si cualquier estado inicial \( X(0) \) puede ser transformado en cualquier estado final \( X(t_f) \) en un tiempo finito \( t_f > 0 \) por un control \( u(t) \). \\ Un sistema es controlable si la matriz de controlabilidad tiene rango completo, es decir, rango $n$ \[ \mathcal{C} = [B, AB, A^2 B, \dots, A^{n-1} B] \]
    \item  Observabilidad: un sistema es observable si cada estado inicial \( X(0) \) puede ser exactamente determinado a partir de mediciones de la salida en un intervalo $0\leq t \leq t_f$. \\ Un sistema es observable si la matriz de observabilidad tiene rango completo, rango n \[ \mathcal{O} = \begin{bmatrix} C \\ CA \\ CA^2 \\ \vdots \\ CA^{n-1} \end{bmatrix} \]

\end{itemize}

\subsection*{Clasificación de sistemas según controlabilidad y observabilidad}

Un sistema puede clasificarse en cuatro subsistemas según su controlabilidad y observabilidad:

\begin{itemize}
    \item \textbf{Completamente controlable y observable}: Todos los estados son controlables y observables.
    \item \textbf{Completamente controlable pero no observable}: Todos los estados son controlables, pero algunos no son observables.
    \item \textbf{No controlable pero observable}: Algunos estados no son controlables, pero todos son observables.
    \item \textbf{No controlable y no observable}: Algunos estados no son controlables ni observables.
\end{itemize}

En algunos casos podemos determinar la Controlabilidad y Observabilidad de un sistema, representando en diagrama de bloques su ecuación de estado, y luego por inspección se determinan los estados desacoplados de la entrada y la salida. Los estados que estén desacoplados de la entrada serán No Controlables y los estados que estén desacoplados de la salida serán No Observables.

\section*{Variables de Fase}

Las variables de fase son una forma canónica de representar un sistema en el espacio de estados, donde las variables de estado se eligen como la salida del sistema y sus derivadas sucesivas.

Para un sistema de orden \( n \) con una entrada \( u(t) \) y una salida \( y(t) \), las variables de fase se definen de la siguiente manera:

\[
    x_1(t) = y(t)
\]

\[
    x_2(t) = \dot{y}(t)
\]

\[
    \vdots
\]

\[
    x_n(t) = y^{(n-1)}(t)
\]

La ecuación diferencial del sistema se puede escribir como:

\[
    y^{(n)} + a_{n-1} y^{(n-1)} + \dots + a_1 \dot{y} + a_0 y = b_m u^{(m)} + \dots + b_0 u
\]

En forma de variables de fase:

\[
    \dot{x}_1 = x_2
\]

\[
    \dot{x}_2 = x_3
\]

\[
    \vdots
\]

\[
    \dot{x}_{n-1} = x_n
\]

\[
    \dot{x}_n = -a_0 x_1 - a_1 x_2 - \dots - a_{n-1} x_n + b_0 u + b_1 \dot{u} + \dots + b_m u^{(m)}
\]

La representación en espacio de estados usando variables de fase es:

\[
    \dot{x} = \begin{bmatrix}
        0      & 1      & 0      & \dots  & 0        \\
        0      & 0      & 1      & \dots  & 0        \\
        \vdots & \vdots & \vdots & \ddots & \vdots   \\
        0      & 0      & 0      & \dots  & 1        \\
        -a_0   & -a_1   & -a_2   & \dots  & -a_{n-1}
    \end{bmatrix} x + \begin{bmatrix}
        0      \\
        0      \\
        \vdots \\
        0      \\
        b_0
    \end{bmatrix} u
\]

\[
    y = \begin{bmatrix}
        1 & 0 & 0 & \dots & 0
    \end{bmatrix} x
\]

Esta representación es útil para el diseño de controladores y observadores.

\subsection*{Caso con ceros en el numerador}

\[G(s) = \frac{K[C_m s^{m-1} + c_{m-1} s^{m-2} + \dots + c_2 s + c_1]}{s^n + a_n s^{n-1} + a_{n-1} s^{n-2} + \dots + a_2 s + a_1}\]

La representación en variables de estados no admite derivadas de la entrada, por ello es necesario reconstruir las ecuaciones en una forma admicible.

\subsubsection{Modificando la matriz C}
\[
    G(s) = \frac{Y(s)}{U(s)} = \frac{X_1(s)}{U(s)} \frac{Y(s)}{X_1(s)} = \frac{K(c_2 s + c_1)}{s^3 + a_3 s^2 + a_2 s + a_1}
\]

donde
\[
    \frac{X_1(s)}{U(s)} = \frac{K}{s^3 + a_3 s^2 + a_2 s + a_1}
\]
\[
    \frac{Y(s)}{X_1(s)} = c_2 s + c_1
\]

la primera ecuación representa el caso sin cero. Mientras que la segunda nos da la siguiente ecuación de la salida

$$
    y(t)  = c_2 \dot{x}_1(t) + c_1 x_1(t)
$$
$$
    y(t)  = c_2 x_2(t) + c_1 x_1(t)
$$


\[
    \begin{bmatrix}
        \dot{x}_1     \\
        \dot{x}_2     \\
        \vdots        \\
        \dot{x}_{n-1} \\
        \dot{x}_n
    \end{bmatrix} =
    \begin{bmatrix}
        0      & 1    & 0    & \cdots & 0        \\
        0      & 0    & 1    & \cdots & 0        \\
        \vdots &      &      & \ddots & \vdots   \\
        0      & 0    & 0    & \cdots & 1        \\
        -a_0   & -a_1 & -a_2 & \cdots & -a_{n-1}
    \end{bmatrix}
    \begin{bmatrix}
        x_1     \\
        x_2     \\
        \vdots  \\
        x_{n-1} \\
        x_n
    \end{bmatrix} +
    \begin{bmatrix}
        0      \\
        0      \\
        \vdots \\
        0      \\
        K
    \end{bmatrix} u(t)
\]

\[
    y(t) =
    \begin{bmatrix}
        c_1 & c_2 & \cdots & c_m & 0 & \cdots & 0
    \end{bmatrix}
    \begin{bmatrix}
        x_1    \\
        x_2    \\
        \vdots \\
        x_n
    \end{bmatrix}
\]
\subsubsection{Modificando la matriz B}
Para hacer eso se utiliza la propiedad de que la función de transferencia es única.

\[ G(s) = C (sI - A)^{-1} B \]


\[
    G(s) = \frac{\begin{bmatrix}
        1 & 0 & 0 & \dots & 0
    \end{bmatrix}Num{[sI - A]^{-1}}\begin{bmatrix}
        b_1   \\
        b_2   \\
        \dots \\
        b_n
    \end{bmatrix}}{s^n + a_n s^{n-1} + a_{n-1} s^{n-2} + \dots + a_2 s + a_1}
    = \frac{K[C_m s^{m-1} + c_{m-1} s^{m-2} + \dots + c_2 s + c_1]}{s^n + a_n s^{n-1} + a_{n-1} s^{n-2} + \dots + a_2 s + a_1}
\]

Se igualan los coeficientes del denominador y se despejan $b_i$.

Las variables de fase son una forma sencilla de representar el sistema, pero tienen una desventaja que no son variables físicas.
\end{document}
